\documentclass[11pt, oneside]{article} 
\usepackage{geometry}
\geometry{letterpaper} 
\usepackage{graphicx}
	
\usepackage{amssymb}
\usepackage{amsmath}
\usepackage{parskip}
\usepackage{color}
\usepackage{hyperref}

\graphicspath{{/Users/telliott/Github/number_theory/png/}}
% \begin{center} \includegraphics [scale=0.4] {gauss3.png} \end{center}

\title{Multiplicative inverse}
\date{}

\begin{document}
\maketitle
\Large

\subsection*{statement}

$a$ is the multiplicative inverse of $b$ (mod $n$) if
\[ ab = 1 \ \text{mod} \ n \]

That is, 
\[ n = kab + 1 \]

Obviously, $a$ and $b$ must be co-prime to $n$.  $n$ cannot divide either $a$ or $b$ because then it must be that $n | 1$.

\subsection*{theorem}

If $p$ is prime, then (mod $p$)
\begin{quote} every $1 < a < p$ has a multiplicative inverse\end{quote}

Furthermore, we claim that \emph{every} product has two unique factors mod $p$.

The statement
\[ ab \equiv r \ \text{mod} \ p \]
is equivalent to
\[ ab = qp + r \]

Suppose that 
\[ ab' = q'p + r \]
then
\[ a(b - b') = (q - q')p \]
\[ a(b - b') \equiv 0 \]
Since $a \ne 0$, it must be true that $b = b'$.

\subsection*{examples}

Suppose $a = 6$ and $b = 10$.  Then, $g = 2$ and so we write Bezout's identity:
\[ 6s + 10t = 2 \]
    
Obviously, one of $s$ or $t$ must be negative.  

One solution is $s = 2, t = -1$.  Another is $s = -3, t = 2$.  

More generally, the integers of the form $sa + tb$ are exactly the multiples of $d$.

Corollary:  $a$ and $b$ are co-prime with gcd $(a,b) = 1$, if and only if there exist integers $s$ and $t$ such that
\[ sa + tb = 1. \]

For example, if $a = 6$ and $b = 5$ then we can find
\[ 6s + 5t = 1 \]
A simple solution is $s = 1, t = -1$.

An application is that if
\[ 6s + 5t = 1 \]
\[ 6s = 1 - 5t \]
    
then (mod $t$):
\[ 6s = 1 \]
$6$ is the multiplicative inverse of $s$ mod $t$.

As an example, construct a table for $p = 17$.  In addition to $1 \cdot 17$, there are seven pairs and one square:

\begin{verbatim}
2.9 =    18 = 1
3.6 =    18 = 1
5.7 =    35 = 1
4.13 =   52 = 1
8.15 =  120 = 1
10.12 = 120 = 1
11.14 = 154 = 1
16.16 = 256 = 1
\end{verbatim}

Notice that $16$ is its own inverse \emph{and} $16 \equiv -1$ mod $17$.  See Wilson's theorem.

\subsection*{uniqueness}

Bezout's lemma says that for every $a$ with gcd$(a,p)$ = 1 (and this is true of every $a < p$, there exists 
\[ sa + tp = 1 \]
That is
\[ sa \equiv 1  \ \text{mod} \ p  \]
So, there exists an $s$ which is the multiplicative inverse for $a$.

Why is there only one such $s$ smaller than $p$?

Suppose that $a$ has two inverses mod $p$.  That is
\[ as \equiv as' \equiv 1 \]
By the cancellation property 
\[ s \equiv s' \]

\end{document}