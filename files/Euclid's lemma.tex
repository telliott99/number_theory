\documentclass[11pt, oneside]{article} 
\usepackage{geometry}
\geometry{letterpaper} 
\usepackage{graphicx}
	
\usepackage{amssymb}
\usepackage{amsmath}
\usepackage{parskip}
\usepackage{color}
\usepackage{hyperref}

\graphicspath{{/Users/telliott/Github/calculus_book/png/}}
% \begin{center} \includegraphics [scale=0.4] {gauss3.png} \end{center}

\title{Euclid's Lemma proof}
\date{}

\begin{document}
\maketitle
\Large

Since we just established the prime factorization theorem independent of Euclid’s lemma, we can use it in a simple proof of the same.

\subsection*{Euclid's lemma}

Suppose that $n = a \cdot b$. If $p|n$ then either $p|a$ or $p|b$ or both.

\subsection*{proof}

Suppose to the contrary, $p|n$ but $p$ does not divide either $a$ or $b$.

Both $a$ and $b$ have a unique prime factorization and those factors multiplied together are the prime factors of $n$. These factors do not include $p$, and yet the factorization is unique. 

This is a contradiction.

$p$ must divide either $a$ or $b$, or both.

$\square$

\subsection*{example}

Note that this is not necessarily true for non-primes.  For example, $6 \cdot 10 = 60 | 4$ but neither $6 | 4$ nor $10 | 4$.  This happens because $2$ is a prime factor of both $6$ and $10$, generating a factor of $4$ in the product.

\end{document}