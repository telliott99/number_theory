\documentclass[11pt, oneside]{article} 
\usepackage{geometry}
\geometry{letterpaper} 
\usepackage{graphicx}
	
\usepackage{amssymb}
\usepackage{amsmath}
\usepackage{parskip}
\usepackage{color}
\usepackage{hyperref}

\graphicspath{{/Users/telliott/Github/number_theory/png/}}
% \begin{center} \includegraphics [scale=0.4] {gauss3.png} \end{center}

\title{Euclidean Algorithm}
\date{}

\begin{document}
\maketitle
\Large

Consider two natural numbers $a$ and $b$.  Usually $a$ is an integer (i.e., it is allowed to be negative), but to keep things simple here we will say that $a$ and $b$ are positive integers, with $a > b$.

We can find their \emph{greatest common divisor}.

This is written gcd$(a,b)$, or just $(a,b)$.

$(a,b)$ is the largest number which evenly divides both $a$ and $b$.

To find it, simply write the unique prime factorization of $a$ and $b$.

\begin{verbatim}
180 = 2.2.3.3.5
140 = 2.2.    5.7
gcd = 2.2.    5     = 20
\end{verbatim}

Pick out the common factors and the gcd$(a,b)$ will be their product.

\subsection*{problem}

However, factorization is hard, and is actually impossible for the really large numbers used in cryptography.  It is important that we do not need to factor $a$ and $b$ for the algorithm to work.

\subsection*{how it works}

Find integers $0 \le r < b$ and $q > 0$ such that
\[ a = b \cdot q + r \]
This relies on the quotient remainder rule.

$\circ$ If $r = 0$ we are done:  $b$ divides $a$ equally.  Otherwise

$ \circ$ switch $a = b$ and $b = r$ and repeat.  Then $b$ is the gcd of the original $a$ and $b$.  

In our example

\begin{verbatim}
180 = 140.1 + 40
140 =  40.3 + 20
 40 =  20.2 + 0
gcd =  20
\end{verbatim}

\subsection*{why it works}

Now let $u$ be the largest integer that divides both $a$ and $b$ (the greatest common divisor)
\[ a = su \]
\[ b = tu \]
Then 
\[ a = b \cdot q + r \]
\[ su = q \cdot tu + r \]
\[ r = su - q \cdot tu \]
We conclude that $u$ divides $r$.

Every common divisor of $a$ and $b$ is also a divisor of $b$ and $r$.

\subsection*{Python code}

Here are two examples of programs in different styles that implement the algorithm (with no error checking):

\begin{verbatim}
def gcd(a,b):
    r = a % b
    if r == 0:
        return b
    return gcd(b,r)
\end{verbatim}

\begin{verbatim}
def gcd(a,b):
    r = a % b
    while r != 0:
        a,b = b,r
        r = a % b
    return b
\end{verbatim}

The first version is \emph{recursive}, it may call itself.  The second uses a \textbf{while} loop to accomplish the same thing.


\end{document}