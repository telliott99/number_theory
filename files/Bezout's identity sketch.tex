\documentclass[11pt, oneside]{article} 
\usepackage{geometry}
\geometry{letterpaper} 
\usepackage{graphicx}
	
\usepackage{amssymb}
\usepackage{amsmath}
\usepackage{parskip}
\usepackage{color}
\usepackage{hyperref}

\graphicspath{{/Users/telliott/Github/number_theory/png/}}
% \begin{center} \includegraphics [scale=0.4] {gauss3.png} \end{center}

\title{Bezout's identity, first look}
\date{}

\begin{document}
\maketitle
\Large

Bezout's identity relies on the fact that Euclid's algorithm works, and it provides information that can be used to write a linear combination:
\[ sa + tb = g \]
where $s$ and $t$ are integers and $g$ is the gcd$(a,b)$.

\subsection*{extended Euclid's algorithm}

If the algorithm is carried out sequentially:

\[ a = b \cdot q_1 + r_1 \ \ \ \ \ \ (0 < r_1 < b) \]
\[ b = r_1 \cdot q_2 + r_2 \ \ \ \ \ \ (0 < r_2 < r_1) \]
\[ r_1 = r_2 \cdot q_3 + r_3 \ \ \ \ \ \ (0 < r_3 < r_2) \]
\[ \dots \]
\[ r_{n} = r_{n+1} \cdot q_{n+2} + 0 = g \ \ \ \ \ \ \]

The successive remainders form a steadily decreasing sequence of positive numbers:
\[ b > r_1 > r_2 > r_3 \dots > r_n > r_{n+1} > 0 \]

After \emph{at most} $b$ steps, and usually much faster, the algorithm must terminate, when say, $r_{n+2}$ is 0.  Then $r_n = (a,b) = g$.

\subsection*{consequence}

We claim that we can find integers $s$ and $t$ such that
\[ r_n = sa + tb \]

Start with $r_1$.  We have that 
\[ r_1 = a - b \cdot q_1 \]  

Change the notation from $q,r$ to $s,t$ with $s_1 = 1$ and $t_1 = -q_1$ so
\[ r_1 = s_1 \cdot a + t_1 \cdot b \]
The next equation is $r_2 = b - q_2 \cdot r_1 $ so we can write
\[ r_2 = b - q_2(s_1 \cdot a + t_1 \cdot b) \]
\[ = -q_2 \cdot s_1 \cdot a + (1 - t_1)b \]

Relabel $s_2 = -q_2 \cdot s_1$ and $t_2 = 1 - t_1$ so
\[ r_2 = s_2 \cdot a + t_2 \cdot b \]
and then just work our way down the (finite) series of equations.
\[ \dots \]
\[ r_{n} = s_{n} \cdot a + t_{n} \cdot b \]
This continues until $r_{n} =  g$, i.e. when $r_{n+1} = 0$.

\[ g = r_n = s_{n} a + t_{n} b \]
\[ g = sa + tb \]

$\square$

The proof can be done more rigorously, but we can see why the claim is correct.

\end{document}