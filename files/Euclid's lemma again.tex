\documentclass[11pt, oneside]{article} 
\usepackage{geometry}
\geometry{letterpaper} 
\usepackage{graphicx}
	
\usepackage{amssymb}
\usepackage{amsmath}
\usepackage{parskip}
\usepackage{color}
\usepackage{hyperref}

\graphicspath{{/Users/telliott/Github/calculus_book/png/}}
% \begin{center} \includegraphics [scale=0.4] {gauss3.png} \end{center}

\title{Euclid's Lemma}
\date{}

\begin{document}
\maketitle
\Large

\subsection*{Euclid's lemma}

Every natural number $n > 1$, i.e. every positive integer greater than $1$, is either prime, or it is the product of two smaller natural numbers $a$ and $b$.

Suppose a given prime $p$ divides $n = ab$, i.e. $p|n$.

We claim that either $p|a$ or  $p|b$ (or both).

Given the theorem on prime factorization, proved by a method independent of Euclid's lemma, there is a very simple proof of Euclid's lemma.

Here is a more standard approach.

\subsection*{Bezout's identity}

We rely on Bezout's identity, which says that there exist integers $r$ and $s$ such that
\[ ra + sp = d \]

where $d$ is the greatest common divisor of $a$ and $p$.

Of course, if $p$ is prime, then
\[ ra + sp = 1 \]

\subsection*{proof}

Suppose that $p|n = ab$ but gcd$(p,a) = 1$.

Then, we can find a linear combination of $a$ and $p$ in the integers such that:
\[ ra + sp = 1 \]

But then, 
\[ b(ra + sp) = b \]
\[ rab + spb = b \]

Since $p|p$ and $p|ab$ (by hypothesis), $p|b$, as desired.

\subsection*{converse}
On the other hand, if $p$ is not prime, it must be composite, i.e. $p=ab$.  

In that case, $p$ divides neither $a$ nor $b$ (since they are smaller than $p$).  

\url{https://artofproblemsolving.com/wiki/index.php/Euclid%27s_Lemma}

\end{document}