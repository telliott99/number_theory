\documentclass[11pt, oneside]{article} 
\usepackage{geometry}
\geometry{letterpaper} 
\usepackage{graphicx}
	
\usepackage{amssymb}
\usepackage{amsmath}
\usepackage{parskip}
\usepackage{color}
\usepackage{hyperref}

\graphicspath{{/Users/telliott/Github/number_theory/png/}}
% \begin{center} \includegraphics [scale=0.4] {gauss3.png} \end{center}

\title{Wilson's Theorem}
\date{}

\begin{document}
\maketitle
\Large

\subsection*{theorem}

Let $p$ be prime.  Then
\[ (p-1)! = -1 \ \text{mod} \ p \]

\subsection*{proof}

If $p$ is prime, then $x < p$ may be its own multiplicative inverse. 
\[ x^2 = 1 \ \text{mod} \ p  \]

However, this (must) happen only if
\[ x^2 - 1 = 0 \ \text{mod} \ p   \]
\[ (x - 1)(x + 1) = 0 \ \text{mod} \ p  \]

That is, $x - 1 = 0$ mod $p$ or $x + 1 = 0$, mod $p$.

But the only numbers in $[1 \dots p]$ which satisfy this are $1$ and $p-1$.  

Therefore, no other number can be its own multiplicative inverse.

Consider the product 
\[ (p-1)! = 1 \cdot 2 \dots \cdot (p-1) \]

For nearly all the factors in the factorial the multiplicative inverse is also present, and that pair then contribute only $1$ to the product (mod $p$).

The exceptions $1$ and $(p-1)$ are their own inverses.  The result follows.



\end{document}