\documentclass[11pt, oneside]{article} 
\usepackage{geometry}
\geometry{letterpaper} 
\usepackage{graphicx}
	
\usepackage{amssymb}
\usepackage{amsmath}
\usepackage{parskip}
\usepackage{color}
\usepackage{hyperref}

\graphicspath{{/Users/telliott/Github/number_theory/png/}}
% \begin{center} \includegraphics [scale=0.4] {gauss3.png} \end{center}

\title{Congruence}
\date{}

\begin{document}
\maketitle
\Large

Here are three equivalent statements about congruence.  Two numbers $a$ and $b$ are congruent modulo $m$ (or with modulus $m$) if

$\circ$ \ they leave the same remainder when divided by $m$.
\[ a = jm + r, \ \ \ \ \ \ \ b = km + r \]
and then we have that
\[ a - b = m(j - k) = dm \]

$\circ$ \ $m$ divides the difference between $a$ and $b$ evenly

so

$\circ$ \ $a = b + dm$

This is written as $m|a-b$ and sometimes as
\[ a \equiv b \ \text{(mod} \ m \]

\subsection*{example}
\[ 7 \equiv 12 \ \text{(mod} \ 5) \]

because $7 = 1 \cdot 5 + 2$ and $12 = 2 \cdot 5 + 2$.  This extends to negative numbers
\[ -8 = -2 \cdot 5 + 2 \]

The numbers evenly divisible by $m = 5$ are 
\[ \dots -10, -5, 0, 5 , 10 \dots \]
because we can find integer $d$ such that $dm = a$ or $dm = b$.

If there is a remainder, for positive $a$, the difference between $a$ and the next \emph{smaller} multiple of $m$ is $r$.

This is also true for $b < 0$, but to avoid confusion remember that the next smaller multiple is \emph{larger} in absolute value.  The multiples of $5$ that "bracket" $-8$ are $-2 \cdot 5$ and $-1 \cdot 5$.  Both these statements are true:
\[ -10 < -5 \]
\[ |-10| > |5| \]
For $a > 0$ and $b < 0$, it is still true that $m$ divides the difference between $a$ and $b$ evenly.  The difference between $12$ and $-8$ is $20$, which \emph{is} evenly divisible by $5$.

Courant and Robbins:

\begin{quote}The usefulness of Gauss’s congruence notation lies in the fact that congruence with respect to a fixed modulus has many of the formal properties of ordinary equality.\end{quote}

For example, congruence is transitive.  If (mod $m$):
\[ a \equiv b, \ \ \ \ \ \ b \equiv c \ \ \  \rightarrow \ \ \ a \equiv c \]

\subsection*{arithmetic}

Congruences may be added, subtracted and multiplied.  Simply consider the remainders modulus $m$.  If (mod $m$):
\[ a \equiv r_1, \ \ \ \ \ \ b \equiv r_2 \]
then
\[  a + b \equiv r_1 + r_2 \]
\[  a - b \equiv r_1 - r_2 \]
\[  ab \equiv r_1 \cdot r_2 \]

where it is recognized that the results $r_1 \pm r_2$ and $r_1 \cdot r_2$ may need to be taken module $m$ again.

Proofs:

We can find $j$ and $k$ such that
\[ a = jm + r_1, \ \ \ \ \ \ \ b = km + r_2 \]
so
\[ a + b = (j + k)m + r_1 + r_2 \]
\[ a - b = (j - k)m + r_1 - r_2 \]
and
\[ a \cdot b = (jm + r_1) + (km + r_2) \]
\[ a \cdot b = m \cdot (jkm + jr_2 + kr_1) + r_1 \cdot r_2 \]

$\square$

Multiplication implies the \emph{cancellation} property.  If 
\[ ac \equiv bc \]
then
\[ a \equiv b \]

Proof:  write the multiplication equality backward.

Also, if $a \equiv a'$ and $b \equiv b'$ then
\[ a + b = a' + b' \]

\subsection*{powers of 10}
Suppose we look at modulus $11$ as an example.
\[ 10 \equiv -1 \]
\[ 10^2 \equiv (-1)(-1) = 1 \]
\[ 10^3 \equiv (1)(-1) = -1 \]
so we obtain alternating plus and minus one.

Therefore, any integer
\[ z = a_0 + a_1 \cdot 10 + a_2 \cdot 10^2 + \dots + + a_n \cdot 10^n \]
leaves the same remainder on division by $11$ as
\[ (a_0 - a_1) + (a_2 - a_3) + \dots \]

It follows that a number is divisible by $11$ if and only if the alternating sum of its digits is divisible by $11$.

\subsection*{example}

To what number between $0$ and $12$ inclusive is the following product congruent modulo $13$?
\[ 3 \cdot 7 \cdot 11\cdot 17 \cdot 19 \cdot 23 \cdot 29 \cdot 113 \]

We have:
\[ 3 \cdot 7 = 21 \equiv 8 \]
\[ 11 \cdot 17 \equiv 11 \cdot 4 = 5 \]
\[ 19 \cdot 23 \equiv 6 \cdot 10 = 60 \equiv 8 \]
\[ 29 \cdot 113 = 3 \cdot 9 = 27 \equiv 1 \]
so
\[ 8 \cdot 5 \equiv 40 \equiv 1 \]
And the final result is $8$.

which is easily checked in Python or by using a calculator to find the product ($5623656423$), and then divide by $13$.

\end{document}