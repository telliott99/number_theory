\documentclass[11pt, oneside]{article} 
\usepackage{geometry}
\geometry{letterpaper} 
\usepackage{graphicx}
	
\usepackage{amssymb}
\usepackage{amsmath}
\usepackage{parskip}
\usepackage{color}
\usepackage{hyperref}

\graphicspath{{/Users/telliott/Github/number_theory/png/}}
% \begin{center} \includegraphics [scale=0.4] {gauss3.png} \end{center}

\title{Fermat's theorem}
\date{}

\begin{document}
\maketitle
\Large


We are concerned not with the famous "last" theorem but with Fermat's \emph{little} theorem.

The theorem says that for any integer $a$ and any prime $p$ \emph{which does not divide a} ($a$ must be \emph{co-prime} to $p$):

\[ a^{p-1} \equiv 1 \ \text{mod} \ p \]
equivalently
\[ a^p \equiv a \ \text{mod} \ p \]

Euler proved this theorem (Fermat did not) and he extended it by showing that it is true, not just for $p$ prime, but for any $n$ coprime to $a$.

\subsection*{examples}

mod $13$:
\[ 2^4 = 16 \equiv 3 \]
\[ 2^{12} = 3 \cdot 3 \cdot 3 = 27 \equiv 1 \]
and, mod $11$
\[ 5^2 = 25 \equiv 3 \]
\[ 5^4 = 9 \]
\[ 5^8 = 81 \equiv 4 \]
\[ 5^{10} = 3 \cdot 4 = 12 \equiv 1 \]

\subsection*{proof}

Write the multiples of $a$ smaller than $pa$.  Let
\[ m_1 = a \]
\[ m_2 = 2a \]
\[ \dots \]
\[ m_{p-1} = (p-1)a  \]

\subsection*{lemma}

$\circ$ \ No two of these, say $m_i$ and $m_j$, can be congruent mod $p$.

Proof:

Suppose on the contrary that there exist $m_j$ and $m_i$ that \emph{are} congruent (take $j > i$).  Congruence means that their difference:
\[ m_j - m_i = (j - i)a \]
is evenly divisible by $p$.  

But by assumption, $p$ does not divide $a$.  

So then $j - i$ must be divisible by $p$ (note:  this is Euclid's lemma).  

But
\[ 1 \le i < j \le (p-1) \] 
Thus, certainly $j - i < p$ and so $p$ cannot divide $j-i$ either.

Therefore, no two $m$'s are congruent.

As a result, $p$ cannot divide the product $1 \cdot 2 \dots \cdot p-1$.

Another way of putting this is to say that no prime $p$ can evenly divide any number smaller than itself, since each of those numbers has its own unique prime factorization, made up of factors smaller than $p$, and $p$ does not evenly divide any of them.

Restatement:  no $m \equiv 0$ mod $p$.

\subsection*{proof of the theorem}

Since there are $p-1$ terms $m$, and no two are congruent mod $p$, they must correspond to $1 \dots (p-1)$ mod $p$, although they may occur in a different order after multiplication by $a$.

The key result is that when these terms $m$ are multiplied out:
\[ 1a \cdot 2a \dots \cdot (p-1)a = (p-1)! \ a^{p-1} \]

after evaluation mod $p$, the same terms are equal to:
\[ 1 \cdot 2 \dots \cdot (p-1)  = (p-1)! \]

So we can cancel the $(p-1)!$, equate the two, and obtain:
\[ a^{p-1} = 1 \ \text{mod} \ p \]
\[ a^{p} = a \ \text{mod} \ p \]

$\square$

The requirement for $a$ and $p$ coprime arises because it ensures that each term $m_i$ appears only once. 


\end{document}