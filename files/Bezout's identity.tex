\documentclass[11pt, oneside]{article} 
\usepackage{geometry}
\geometry{letterpaper} 
\usepackage{graphicx}
	
\usepackage{amssymb}
\usepackage{amsmath}
\usepackage{parskip}
\usepackage{color}
\usepackage{hyperref}

\graphicspath{{/Users/telliott/Github/number_theory/png/}}
% \begin{center} \includegraphics [scale=0.4] {gauss3.png} \end{center}

\title{Bezout's identity}
\date{}

\begin{document}
\maketitle
\Large

This proof is a bit challenging at the critical step.  

We follow Aitken 

\url{https://public.csusm.edu/aitken_html/m422/Handout1.pdf}

Another source is Hefferon 

\url{http://joshua.smcvt.edu/numbertheory/book.pdf} 

\subsection*{theorem}

$\circ$ \ let $a$ and $b$ be integers, not both zero

$\circ$ \ form a linear combination: $sa + tb$

$\circ$ \ there is a least positive such combination with value $d$

$\circ$ \ there is also a greatest common divisor or gcd$(ab) = g$

$\circ$ \ the theorem says that $g = d$

\subsection*{proof of the theorem}

Def 1.  A common divisor of $a$ and $b$ is an integer that divides both $a$ and $b$.

Def 2.  A linear combination is $sa + tb$, for integer $s$ and $t$

P3.  Let $d$ be a common divisor of $(a,b)$, then $d \ | \ sa + tb$.

Proof:

We have $a = dk$ and $b =dl$ for some integer $k$ and $l$ so

\[ sa + tb = s(dk) + t(dl) = d(sk + tl) \]

P4.  If $a$ and $b$ are both non-zero, then the GCD exists.

Proof:  

Suppose $a \ne 0$ and $S$ is the set of common divisors.  $1 \in S$.
    
$\circ$ \ $S$ is not empty since $1 \in S$

$\circ$ \ the maximum element of $S$ is $|a|$

$\circ$ \ Therefore $S$ has a maximum $d$ and $d \ge 1$

P5.  There is a \emph{least} positive integer combination $sa + tb$.

Proof:  For convenience suppose $a \ne 0$.  Let $S$ be the set of positive linear combinations.

$\circ$ \ clearly, $|a| \in S$

$\circ$ \ $S$ is not empty

$\circ$ \ $S$ has a minimum

Here is the tricky part.

\subsection*{lemma}

If $a \ne 0$ and $b \ne 0$, then the least positive linear combination of $a$ and $b$ is a common divisor of $a$ and $b$.

Proof:  

Let $m = sa + tb$ be the least positive linear combination of $a$ and $b$.

Using the Quotient Remainder Rule write $a = qm + r$.  So

\[ r = a - qm \]
\[ = a - q(sa + tb) \]
\[ = a(1 - qs) - qtb \]

The quotient rule defines $0 \le r < m$.

$\circ$ \ r is non-negative

$\circ$ \ $r$ is a linear combination (above)

$\circ$ \ but $m$ is the smallest positive linear combination

Therefore $r = 0$.

Since $r = 0$, $a = qm$ and therefore $m|a$.  

Similarly, $m|b$.  $m$ is a common divisor of $a$ and $b$.

\subsection*{theorem}

T7.  (Bezout's Identity).  If $a$ and $b$ are not both zero, then the least positive linear combination of $a$ and $b$ is equal to their greatest common divisor.

Proof:

Let $m$ be the least positive linear combination, and let $g$ be the GCD.

$\circ$ \  $g|m$ by Proposition 3, which means that $g \le m$

$\circ$ \  by the lemma, $m$ is a common divisor

$\circ$ \  the greatest common divisor is $g$, so $g < m$ cannot be true

Therefore, $g = m$.

\end{document}